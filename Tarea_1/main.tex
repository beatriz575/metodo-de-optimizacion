\documentclass[12pt]{article}
\usepackage[utf8]{inputenc}
\usepackage[spanish]{babel}
\usepackage{graphicx}
\usepackage{geometry}
\geometry{left=3cm, right=3cm, top=3cm, bottom=3cm}

\begin{document}

\begin{titlepage}
    \begin{center}
    
        
        \begin{minipage}{0.45\textwidth}
            \centering
            \includegraphics[width=3.5cm]{LOGOUNAP.jpg} 
        \end{minipage}
        \hfill
        \begin{minipage}{0.45\textwidth}
            \centering
            \includegraphics[width=3.5cm]{LOGOFINESI.jpg} 
        \end{minipage}
        
        \vspace{1cm}
        
    
        {\Large \textsc{UNIVERSIDAD NACIONAL DEL ALTIPLANO - PUNO}}\\[0.5cm]
        {\large \textsc{ESCUELA PROFESIONAL DE INGENIERÍA ESTADÍSTICA E INFORMÁTICA}}\\[0.4cm]
        {\large \textsc{MÉTODOS DE OPTIMIZACIÓN}}\\[1.5cm]
        
        
        \rule{\linewidth}{0.5mm} \\[0.4cm]
        {\huge \bfseries Definiciones (Variable, Función y Restricción)}\\[0.4cm]
        \rule{\linewidth}{0.5mm} \\[1.5cm]
        
    
        \begin{flushleft}
            \textbf{NOMBRE:} Beatriz Umiña Machaca \\
            \textbf{CÓDIGO:} 230035 \\
            \textbf{DOCENTE:} Fred Torres Cruz \\
            \textbf{CICLO:} V \\
            \textbf{GRUPO:} A \\
        \end{flushleft}
        
        \vfill
        \vfill
\textsc{Puno, 13 de abril de 2025}

    
    
        
    \end{center}
\end{titlepage}

\section*{Introducción}
Cuando se realiza un trabajo de investigación o se analiza un problema real, es importante definir claramente conceptos clave como las \textbf{variables}, las \textbf{funciones} y las \textbf{restricciones}. Estos elementos ayudan a organizar la información y permiten representar de manera precisa lo que se desea analizar o mejorar.

En este caso, tomaremos como ejemplo la variable \textbf{nivel de satisfacción del cliente} en un servicio de entrega a domicilio (delivery), para explicar cada concepto.

\section*{1. Variable}
Una \textbf{variable} es una característica que puede tener diferentes valores. Estas pueden ser cualitativas (categorías) o cuantitativas (números).

\subsection*{Variable seleccionada: Nivel de satisfacción del cliente}
Esta variable mide la percepción del cliente sobre el servicio recibido. En un sistema de delivery, se puede representar con una escala del 1 al 5 (de muy insatisfecho a muy satisfecho).

\subsection*{Organización y operacionalización de la variable}

\begin{center}
\renewcommand{\arraystretch}{1.5} % Espaciado entre filas
\begin{tabular}{|p{4.5cm}|p{6.5cm}|p{4cm}|}
\hline
\textbf{Variable} & \textbf{Definición operativa} & \textbf{Valores asignados} \\
\hline
Nivel de satisfacción del cliente & 
Percepción medida del cliente respecto al servicio recibido en el sistema de delivery, evaluada mediante una escala ordinal del 1 al 5. & 
\begin{itemize}
    \item 1: Muy insatisfecho
    \item 2: Insatisfecho
    \item 3: Regular
    \item 4: Satisfecho
    \item 5: Muy satisfecho
\end{itemize} \\
\hline
\end{tabular}
\end{center}

\subsection*{Ejemplo práctico}
Una empresa de delivery aplica una encuesta al finalizar cada pedido, donde el cliente califica el servicio del 1 al 5. Con estos datos, la empresa puede identificar puntos débiles y mejorar su atención al cliente.

\section*{2. Función}
Una \textbf{función} establece cómo una variable depende de otras. En este caso, la satisfacción del cliente puede depender del tiempo de entrega, la calidad del producto y el costo del servicio.

\begin{center}
\textbf{Satisfacción del cliente} = f(Tiempo de entrega, Calidad del producto, Costo del servicio)
\end{center}

\section*{3. Restricciones}
Las \textbf{restricciones} son condiciones que deben cumplirse en un modelo. Ayudan a establecer los límites dentro de los cuales las variables pueden operar.

\subsection*{ restricciones en nuestro caso practico}
\begin{itemize}
    \item El tiempo de entrega no debe superar los 30 minutos: \textbf{T $\leq$ 30}
    \item El costo del envío no debe ser mayor a 10 soles: \textbf{C $\leq$ 10}
\end{itemize}

Estas restricciones permiten garantizar un servicio eficiente y accesible para los clientes.

\section*{Conclusión}
Comprender las variables, funciones y restricciones es esencial para analizar situaciones reales. En este caso, analizar la satisfacción del cliente ayuda a mejorar servicios de delivery mediante decisiones basadas en datos.

\section*{Bibliografía}
\begin{itemize}
    \item Hernández Sampieri, R. (2014). \textit{Metodología de la investigación}. McGraw-Hill.
    \item Taha, H. A. (2004). \textit{Investigación de operaciones}. Pearson Educación.
    \item Kotler, P., \& Keller, K. L. (2012). \textit{Dirección de marketing}. Pearson Educación.
    \item Pita Fernández, S., & Pértega Díaz, S. (1997). Relación entre variables cuantitativas. Cad Aten Primaria, 4, 141-4.
\end{itemize}


\end{document}

