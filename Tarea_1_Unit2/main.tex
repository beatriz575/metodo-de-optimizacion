\documentclass{beamer}
\usetheme{Madrid} % Puedes cambiar el tema si deseas

\title{Método de Lagrange para Optimización}
\author{Paricela Yana Jimena\\ Butron Maquera Tania Karin \\ Umiña Machaca Beatriz}
\date{\today}

\begin{document}

\frame{\titlepage}

% Diapositiva 1 - ¿Qué es?
\begin{frame}{¿Qué es el Método de Lagrange?}
\begin{itemize}
    \item Técnica matemática para hallar máximos o mínimos de una función.
    \item Se aplica cuando hay restricciones de igualdad: \textit{optimización con restricciones}.
    \item Busca puntos donde la función objetivo es extrema bajo una condición.
\end{itemize}
\end{frame}

% Diapositiva 2 - Definición formal
\begin{frame}{Definición Formal}
\textbf{Problema:}
\begin{block}{}
Maximizar o minimizar $f(x, y, \dots)$ \\
sujeto a $g(x, y, \dots) = 0$
\end{block}

\textbf{Función Lagrangiana:}
\[
\mathcal{L}(x, y, \lambda) = f(x, y) + \lambda (g(x, y))
\]
\textbf{Condición de optimalidad:}
\[
\nabla f = \lambda \nabla g
\]
\end{frame}

% Diapositiva 3 - Pasos
\begin{frame}{Pasos del Método de Lagrange}
\begin{enumerate}
    \item Identificar $f(x, y)$ y $g(x, y) = 0$
    \item Construir $\mathcal{L}(x, y, \lambda) = f(x, y) + \lambda g(x, y)$
    \item Calcular derivadas parciales: 
    \[
    \frac{\partial \mathcal{L}}{\partial x} = 0,\quad \frac{\partial \mathcal{L}}{\partial y} = 0,\quad \frac{\partial \mathcal{L}}{\partial \lambda} = 0
    \]
    \item Resolver el sistema para hallar \( x, y, \lambda \)
    \item Evaluar \( f(x, y) \) en los puntos encontrados
\end{enumerate}
\end{frame}


% Diapositiva 5 - Aplicaciones
\begin{frame}{Aplicaciones del Método de Lagrange}
\begin{itemize}
    \item \textbf{Economía:} maximizar utilidades con presupuestos.
    \item \textbf{Ingeniería:} minimizar costos bajo restricciones técnicas.
    \item \textbf{Física:} se usa en el principio de mínima acción.
    \item \textbf{Machine Learning:} regularización y optimización de funciones.
\end{itemize}
\end{frame}



\begin{frame}{Ejemplo práctico - Enunciado}
\textbf{Problema:} Maximizar la función objetivo:
\[
f(x, y) = xy
\]
\textbf{Sujeto a la restricción:}
\[
x + y = 10
\]

Queremos encontrar los valores de $x$ y $y$ que \textbf{maximicen el producto} $xy$ sin dejar de cumplir la condición de que $x + y = 10$.
\end{frame}
\begin{frame}{Paso 1: Lagrangiana}
Usamos el método de Lagrange para incorporar la restricción a la función objetivo:

\textbf{Lagrangiana:}
\[
\mathcal{L}(x, y, \lambda) = xy + \lambda (10 - x - y)
\]

\textbf{¿Qué significa esto?}

- $xy$ es la función que queremos maximizar.
- $\lambda$ es el multiplicador de Lagrange.
- $(10 - x - y)$ es la forma reordenada de la restricción.

Así, convertimos un problema con restricción en uno sin restricción.
\end{frame}
\begin{frame}{Paso 2: Derivadas parciales}
Derivamos la Lagrangiana con respecto a cada variable:

\[
\frac{\partial \mathcal{L}}{\partial x} = y - \lambda = 0
\]
\[
\frac{\partial \mathcal{L}}{\partial y} = x - \lambda = 0
\]
\[
\frac{\partial \mathcal{L}}{\partial \lambda} = 10 - x - y = 0
\]

\textbf{¿Qué obtenemos?}

Un sistema de tres ecuaciones con tres incógnitas: $x$, $y$ y $\lambda$.
\end{frame}
\begin{frame}{Paso 3: Resolviendo el sistema}
Tenemos el sistema:

\[
y - \lambda = 0 \Rightarrow y = \lambda
\]
\[
x - \lambda = 0 \Rightarrow x = \lambda
\]
\[
10 - x - y = 0 \Rightarrow x + y = 10
\]

\textbf{Sustituyendo:}

Como $x = \lambda$ y $y = \lambda$, entonces:

\[
x + y = 2\lambda = 10 \Rightarrow \lambda = 5
\Rightarrow x = 5, y = 5
\]
\end{frame}
\begin{frame}{Resultado final}
Hemos encontrado el punto óptimo:

\[
x = 5, \quad y = 5
\]

\textbf{Valor máximo:}
\[
f(5, 5) = 5 \times 5 = 25
\]

\textbf{Interpretación:}  
El producto $xy$ es máximo cuando $x$ y $y$ valen ambos 5, cumpliendo la restricción $x + y = 10$.
\end{frame}



\end{document}